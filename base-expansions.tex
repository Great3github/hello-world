\documentclass[a5paper, fleqn]{article}
\usepackage[margin=1cm]{geometry}
\setlength{\footskip}{0.5cm}
\usepackage[utf8]{inputenc}
\usepackage{amsmath}
\usepackage{amsfonts}
\usepackage{fp}
\usepackage{lineno}

\usepackage{xcolor}
\pagecolor[HTML]{0f172a}
\color[HTML]{e2e8f0}

\DeclareMathOperator{\half}{half}
\DeclareMathOperator{\Div}{div}
\DeclareMathOperator{\Mod}{mod}
\DeclareMathOperator{\base}{base}
\newcommand{\nat}{\mathbb{N}}
\newcommand{\posint}{\mathbb{Z}^+}

\begin{document}

\section*{Recursively defining base $b$ expansion calculators}

Our CSE 20 professor defined a bunch of procedures in pseudocode using rather
imperative constructs such as \texttt{while} loops, even though it could be
defined using some epic iterative recursion instead.

\vspace{1em}

\begin{linenumbers}
  \textbf{procedure} $half(n: \text{a positive integer})$

  $r := 0$

  \textbf{while} $n > 1$

  \vspace{-1em}
  \begin{quote}
    $r := r + 1$ \\
    $n := n - 2$
  \end{quote}
  \vspace{-1em}

  \textbf{return} $r$ \{$r$ holds the result of the operation\}

\end{linenumbers}

\vspace{1em}

This could instead be defined with a piecewise function.
\begin{align*}
  \half'       & : \nat \times \nat \to \nat               \\
  \half'(r, n) & = \begin{cases}
                     \half'(r + 1, n - 2) & \text{if } n > 1 \\
                     r                    & \text{otherwise}
                   \end{cases} \\
  \\
  \half        & : \nat \to \nat                           \\
  \half(n)     & = \half'(0, n)
\end{align*}

I'm going to generalize $\half$ to any divisor. It'll be the equivalent to the
\textbf{div} operator, which does integer division.
\begin{align*}
  \Div'          & : \nat \times \nat \times \posint \to \nat    \\
  \Div'(q, n, d) & = \begin{cases}
                       \Div'(q + 1, n - d, d) & \text{if } n \ge d \\
                       q                      & \text{otherwise}
                     \end{cases} \\
  \\
  \Div           & : \nat \times \posint \to
  \nat                                                           \\
  \Div(n, d)     & = \Div'(0, n, d)
\end{align*}

Because $n = dq + r = d(n \;\textbf{div}\; d) + (n \;\textbf{mod}\; d)$, let's
define a $\Mod$ function.
\begin{align*}
  \Mod       & : \nat \times \posint \to
  \nat                                   \\
  \Mod(n, d) & = n - d \cdot \Div(n, d)
\end{align*}

That's pretty poggers. My professor defined a procedure that calculates the
integer part of $\log_b$. As another epic piecewise function,
\begin{align*}
  B              & = \left\{ b \in \posint \,\middle|\, b > 1 \right\}  \\
  \\
  \log'          & : \nat \times \nat \times B \to \nat                 \\
  \log'(r, n, b) & = \begin{cases}
                       \log'(r + 1, \Div(n, b), b) & \text{if } n > b - 1 \\
                       r                           & \text{otherwise}
                     \end{cases} \\
  \\
  \log           & : \nat \times B \to \nat                             \\
  \log(n, b)     & = \log'(0, n, b)
\end{align*}

Before I can convert numbers to bases, I'll first define the set of all base $b$
expansions of a natural number.

First, let $D_b = \left\{ d \in \nat \,\middle|\, d < b \right\}$, the set of
digits available. $D_b \subset E_b$. Then if $e \in E_b$, $d \in D_b$, and $e
  \neq 0$, then $e \circ d \in E_b$. Also, for convenience, let $E_b \subset E_b'$
and $\lambda \in E_b'$; in other words, $E_b'$ is $E_b$ but with the empty
string.

Finally, I think I'll implement the ``Least significant first'' algorithm for
calculating the base $b$ expansion from the right.
\begin{align*}
  \base'          & : E_b' \times \nat \times B \to E_b           \\
  \base'(a, q, b) & = \begin{cases}
                        \base'(\Mod(q, b) \circ a, \Div(q, b), b) &
                        \text{if } q \neq 0                         \\
                        a                                         &
                        \text{otherwise}
                      \end{cases} \\
  \\
  \base           & : \nat \times B \to E_b                       \\
  \base(n, b)     & = \begin{cases}
                        0                     & \text{if } n = 0 \\
                        \base'(\lambda, n, b) & \text{otherwise}
                      \end{cases}
\end{align*}

\dots I think? CSE 20 is fun.

\hfill{$\square$}

\end{document}
